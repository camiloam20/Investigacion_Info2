\documentclass[12pt]{article}
\usepackage{UF_FRED_paper_style}
\setlength{\droptitle}{-5em} 
\onehalfspacing
\title{Los hilos en los microprocesadores}


\date{8 de julio de 2020}
\author{Camilo Alvarez Muñoz\\
\href{mailto:camilo.alvarezm@udea.edu.co}{\texttt{camilo.alvarezm@udea.edu.co}}}
\begin{document}
{\setstretch{.8}
\maketitle}
Nuestras computadoras forman parte fundamental de nuestra vida cotidiana, son necesarias para realizar una gran cantidad de tareas, y conforme nuestra sociedad ha cambiado, también han cambiados nuestras necesidades, los procesadores han tenido que evolucionar para ser capaces de cumplir esas tareas. Los procesadores son los cerebros de nuestros computadores, son aquellos que permiten a los usuarios hacer muchas cosas, como editar fotos, leer un documento, ejecutar un programa o jugar un videojuego. 

\medskip
\bibliography{references.bib} 
\end{document}