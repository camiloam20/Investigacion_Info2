\documentclass[12pt]{article}
\usepackage{UF_FRED_paper_style}
\setlength{\droptitle}{-5em} 
\onehalfspacing
\title{Los hilos en los microprocesadores}


\date{8 de julio de 2020}
\author{Camilo Alvarez Muñoz\\
\href{mailto:camilo.alvarezm@udea.edu.co}{\texttt{camilo.alvarezm@udea.edu.co}}}
\begin{document}
{\setstretch{.8}
\maketitle}
Nuestras computadoras forman parte fundamental de nuestra vida cotidiana, son necesarias para realizar una gran cantidad de procesos, y conforme nuestra sociedad ha cambiado, también han cambiados nuestras necesidades, los procesadores han tenido que evolucionar para ser capaces de cumplir cualquier tipo de  tarea o necesidad que nos sea requerida. Los procesadores son los cerebros de nuestros computadores, son aquellos que permiten a los usuarios hacer muchas cosas, como editar fotos, leer un documento, ejecutar un programa o jugar un videojuego. 

Un procesador se puede descomponer en subprocesadores, o conocidos comúnmente como núcleos (Cores), estos núcleos son unas unidades de procesamiento que nos permiten realizar las instrucciones que requiera procesar el computador. Anteriormente se usaban procesadores con solo un núcleo, pero debido a la cantidad de tareas y necesidades que han surgido con el tiempo, se vio la necesidad de crear procesadores con muchos más núcleos para poder procesar distintas tareas simultáneamente y así aumentar por mucho el rendimiento.

Los núcleos reciben las tareas a realizar, pero, una parte fundamental y básica del procesador son los hilos que en realidad son los principales encargados de esa sensación de simultaneidad al momento de realizar bastantes tareas. Los hilos son flujos o conjuntos de instrucciones que son necesarias para hacer una tarea. Para hacer una gran tarea, el núcleo divide todo el proceso en tareas más pequeñas que se ejecutan de los hilos en forma alternada, al hacer esto simula la sensación de estar realizando tareas al mismo tiempo. Trabajar con hilos permite mejorar los tiempos de espera entre tareas y tener una forma más eficiente y rápida de realizar distintos procesos a la vez.

Existen dos tipos de hilos, los hilos manejado a nivel de usuario, y los hilos manejados a nivel de hardware (Kernel). 

\medskip
\bibliography{references.bib} 
\end{document}