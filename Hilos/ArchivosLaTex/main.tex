\documentclass[12pt]{article}
\usepackage{UF_FRED_paper_style}
\setlength{\droptitle}{-5em} 
\onehalfspacing
\title{Los hilos en los microprocesadores}


\date{8 de julio de 2020}
\author{Camilo Alvarez Muñoz\\
\href{mailto:camilo.alvarezm@udea.edu.co}{\texttt{camilo.alvarezm@udea.edu.co}}}
\begin{document}
{\setstretch{.8}
\maketitle}
Nuestras computadoras forman parte fundamental de nuestra vida cotidiana, son necesarias para realizar una gran cantidad de procesos, y conforme nuestra sociedad ha cambiado, también han cambiados nuestras necesidades, los procesadores han tenido que evolucionar para ser capaces de cumplir cualquier tipo de  tarea o necesidad que nos sea requerida. Los procesadores son los cerebros de nuestros computadores, son aquellos que permiten al computador procesar y manejar los datos e instrucciones que dejan a los usuarios hacer muchas cosas, como editar fotos, leer un documento, ejecutar un programa o jugar un videojuego. 

Un procesador se puede descomponer en subprocesadores, o conocidos comúnmente como núcleos (Cores), estos núcleos son unas unidades de procesamiento que nos permiten ejecutar y procesar las instrucciones que requiera el computador para realizar cierta tarea (Como utilizar un programa, escribir o escuchar un video). Anteriormente se usaban procesadores con solo un núcleo, pero debido a la cantidad de tareas y necesidades que han surgido con el tiempo, se vio la necesidad de crear procesadores como muchos más núcleos para poder procesar distintos programas simultáneamente y así aumentar por mucho el rendimiento \cite{Nucleos_e_Hilos}.
Los núcleos reciben las tareas a realizar pero una parte fundamental y básica del procesador son los hilos que en realidad son los principales encargados de esa sensación de simultaneidad al momento de realizar bastantes tareas. 
Los hilos son secuencias o flujos de instrucciones que son necesarias para realizar un proceso. Para hacer una gran tarea, el procesador divide todo  en tareas más pequeñas (hilos) que son pasadas al núcleo para ser ejecutadas de forma alternada, al hacer esto simula la sensación de estar realizando tareas al mismo tiempo. Trabajar con hilos permite mejorar los tiempos de espera entre tareas, tener una forma más eficiente y rápida de realizar distintos procesos a la vez y un mejor manejo de recursos \cite{Diferencias} .

Existen dos tipos de hilos, los hilos manejado a nivel de usuario (ULT), y los hilos manejados a nivel de Kernel (KLT).
Los hilos a nivel de usuario son aquellos en el que la gestión de hilos es realizada por los programas, entonces el Kernel en ningún momento tiene noción de la existencia de estos  hilos, se usan unas librerías o bibliotecas que nos permiten utilizar muchas funciones destinadas a la planificación y control de hilos, como crear hilos, destruirlos, crear preferencias, mandar mensajes y modificar datos, entre otros. Este tipo de hilos se pueden ejecutar en cualquier sistema operativo y no necesita los privilegios del Kernel, los ULT son mas faciles y rapidos de usar pero tambien tienden a tener algunas desventajas como el uso de multihilos.

Los hilos a nivel de Kernel son aquellos únicamente manejados por el sistema operativo, entonces el es el encargado de gestionar cosas como la distribución de los procesos en los núcleos,  la cantidad de hilos usados, los privilegios y demás opciones. Este manejo de hilos presenta algunas ventajas como la posibilidad de tener múltiples hilos de un mismo proceso en distintos núcleos y la re planificación de hilos bloqueados.


\medskip
\bibliography{references.bib} 
\end{document}