\documentclass[11pt]{article}
\usepackage{UF_FRED_paper_style}
\setlength{\droptitle}{-5em} 
\onehalfspacing
\title{Las interrupciones en los microprocesadores}

\author{Camilo Alvarez Muñoz}
\date{24 de junio de 2020}
\begin{document}
{\setstretch{.8}
\maketitle}
Conforme ha avanzado la tecnología, la humanidad ha sido capaz de resolver tareas que antes requerían de mucho esfuerzo y tiempo, pero ahora, podemos hacer estas tareas de una forma muy sencilla con la ayuda de los avances tecnológicos. Uno de estos avances son los microprocesadores, pequeños dispositivos que nos permiten realizar una infinidad de operaciones lógicas.

Con el tiempo se han ido mejorando los microprocesadores y microcontroladores y entonces se vio la necesidad de introducir un concepto importante, las interrupciones, este concepto es necesario para que el microprocesador o microcontrolador pueda interactuar correctamente con el exterior sin perder su eficiencia y aprovechando su máximo potencial “Una interrupción consiste en un mecanismo por el cual un evento interno o externo puede interrumpir la ejecución de un programa en cualquier momento. A partir de entonces se produce automáticamente un salto a una subrutina de atención a la interrupción, ésta atiende inmediatamente el evento y retoma luego la ejecución del programa exactamente donde estaba en el momento de ser interrumpido, continuando su tarea justo donde la dejó”\cite{apaza2017microcontroladores}. Este concepto, en el caso de los microprocesadores, significa que cuando los periféricos (mouse, teclado, pantalla, entre otros) requieren mandar algún tipo de información o realizar una operación en el microprocesador, esta interrupción se puede atender en cualquier momento de una forma eficiente y que al momento de terminar el evento, se retome las instrucciones que antes se estaban ejecutando.

El concepto de interrupciones es muy importante y puede ser ilustrado por muchos ejemplos de nuestra vida cotidiana, como cuando somos distraídos de lo que estamos haciendo por una llamada telefónica o mensaje y debemos responder para poder retomar lo que estábamos haciendo, o cuando necesitamos recibir algún paquete o abrirle la puerta a alguna persona, los casos son muchos, el punto es que tanto nosotros como los microprocesadores o los microcontroladores estamos expuestos a situaciones exteriores a nuestra forma de operar básica que debemos atender rápidamente.

Se puede hablar de que hubo una evolución “histórica” en las interrupciones, cuando se crearon los primeros microcontroladores en el año 1971 fue introducido un método llamado “polling”, este método consistía en un mecanismo que tenía la instrucción de estar periódicamente monitoreando las entradas/salidas (Periféricos) del microcontrolador verificando si había algún evento que debía ser atendido, este método tuvo que ser reemplazado, ya que no era eficiente, se consumían mucho tiempo y recursos para que el microcontrolador estuviera siempre revisando si ocurría un evento y no permitía al controlador poder enfocarse totalmente en otras tareas y mucha información se podía llegar a perder debido a los periodos de tiempo de monitoreo\cite{reyes2015arduino}.

Debido a la poca eficiencia del “Polling” se vio la necesidad de introducir un nuevo método más directo y eficiente, las interrupciones, el microcontrolador/microprocesador cuenta con un pequeño dispositivo llamado Unidad de Interrupciones, esta unidad es la que recibirá las señales que provienen de los periféricos que indican que tiene que ocurrir una interrupción, la unidad de interrupciones establece prioridad entre cual interrupción atender primero, cuando se recibe una petición de interrupción el procesador pausa las instrucciones que estaba haciendo, ejecuta una rutina de instrucciones previamente establecidas para la interrupción y luego retoma su proceso normal en donde lo había dejado previamente.
\medskip
\bibliography{references.bib} 
\end{document}