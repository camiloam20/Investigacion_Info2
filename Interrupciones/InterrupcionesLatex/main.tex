\documentclass[11pt]{article}
\usepackage{UF_FRED_paper_style}
\onehalfspacing
\setlength{\droptitle}{-5em} %% Don't touch
\title{Las interrupciones en los microprocesadores}

\author{Camilo Alvarez Muñoz}
\date{24 de junio de 2020}
\begin{document}
{\setstretch{.8}
\maketitle}
Conforme ha avanzado la tecnología, la humanidad ha sido capaz de resolver tareas que antes requerían de mucho esfuerzo y tiempo, pero ahora, podemos hacer estas tareas de una forma muy sencilla con la ayuda de los avances tecnológicos. Uno de estos avances son los microprocesadores, pequeños dispositivos que nos permiten realizar una infinidad de operaciones lógicas.

Con el tiempo se han ido mejorando los microprocesadores y microcontroladores y entonces se vio la necesidad de introducir un concepto importante, las interrupciones, este concepto es necesario para que el microprocesador o microcontrolador pueda interactuar correctamente con el exterior sin perder su eficiencia y aprovechando su máximo potencial “Una interrupción consiste en un mecanismo por el cual un evento interno o externo puede interrumpir la ejecución de un programa en cualquier momento. A partir de entonces se produce automáticamente un salto a una subrutina de atención a la interrupción, ésta atiende inmediatamente el evento y retoma luego la ejecución del programa exactamente donde estaba en el momento de ser interrumpido, continuando su tarea justo donde la dejó”\cite{apaza2017microcontroladores}. Este concepto, en el caso de los microprocesadores, significa que cuando los periféricos (mouse, teclado, pantalla, entre otros) requieran mandar algún tipo de información o realizar una operación en el microprocesador, esta interrupción se puede atender en cualquier momento de una forma eficiente y que al momento de terminar el evento, se retome las instrucciones que antes se estaban ejecutando.

Se puede hablar de que hubo una evolución “histórica” en las interrupciones, primero fue introducido un método llamado “polling”, este método consiste en que el microcontrolador o el microprocesador revisa periódicamente todos los pines de entrada/salida para verificar si algún evento fue realizado.

\medskip
\bibliography{references.bib} 
\end{document}