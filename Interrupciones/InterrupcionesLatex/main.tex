\documentclass[11pt]{article}
\usepackage{UF_FRED_paper_style}
\onehalfspacing
\setlength{\droptitle}{-5em} %% Don't touch
\title{Las interrupciones en los microprocesadores}

\author{Camilo Alvarez Muñoz}
\date{24 de junio de 2020}
\begin{document}
{\setstretch{.8}
\maketitle}
Conforme ha avanzado la tecnología, la humanidad ha sido capaz de resolver tareas que antes requerían de mucho esfuerzo y tiempo, pero ahora, podemos hacer estas tareas de una forma muy sencilla con la ayuda de los avances tecnológicos. Uno de estos avances son los microprocesadores.

Con el tiempo se fueron mejorando los microprocesadores y los microcontroladores y se vio la necesidad de introducir un concepto importante, las interrupciones, este concepto es necesario para que el microprocesador o microcontrolador puede interactuar correctamente con el exterior sin perder su eficiencia y aprovechar su máximo potencial “Una interrupción consiste en un mecanismo por el cual un evento interno o externo puede interrumpir la ejecución de un programa en cualquier momento. A partir de entonces se produce automáticamente un salto a una subrutina de atención a la interrupción, ésta atiende inmediatamente el evento y retoma luego la ejecución del programa exactamente donde estaba en el momento de ser interrumpido, continuando su tarea justo donde la dejó”\cite{apaza2017microcontroladores}.

\medskip
\bibliography{references.bib} 
\end{document}